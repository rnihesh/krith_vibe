\documentclass{article}
\usepackage[utf8]{inputenc}
\usepackage{amsmath, amssymb}
\usepackage{graphicx}
\usepackage{hyperref}

\title{Semantic Entropy File System: Technical Report}
\author{SEFS Research Team}
\date{February 2026}

\begin{document}
\maketitle

\begin{abstract}
The Semantic Entropy File System (SEFS) introduces a novel approach to automatic
file organization using semantic embeddings and hierarchical density-based clustering.
Files are embedded into a high-dimensional vector space using transformer models,
then clustered using HDBSCAN to discover natural groupings. This paper describes
the system architecture, embedding strategies, and cluster naming algorithms.
\end{abstract}

\section{Introduction}
Traditional file systems organize data hierarchically based on user-defined folder
structures. SEFS proposes an automated alternative where files are grouped by
semantic similarity rather than manual categorization.

\section{Methodology}
\subsection{Embedding Generation}
Given a document $d$, we compute its embedding vector:
\[ \mathbf{e}_d = f_\theta(d) \in \mathbb{R}^{768} \]
where $f_\theta$ is a pre-trained transformer model (nomic-embed-text).

\subsection{Clustering}
We apply HDBSCAN with cosine distance:
\[ d_{cos}(\mathbf{a}, \mathbf{b}) = 1 - \frac{\mathbf{a} \cdot \mathbf{b}}{|\mathbf{a}||\mathbf{b}|} \]

The minimum cluster size is set to 3 to avoid singleton clusters.

\section{Results}
Preliminary evaluation on a corpus of 500 files shows:
\begin{itemize}
    \item Average cluster purity: 87.3\%
    \item Silhouette score: 0.62
    \item User satisfaction rating: 4.2/5
\end{itemize}

\end{document}
